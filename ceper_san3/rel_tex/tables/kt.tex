
\begin{table}[H] \centering 

\begin{minipage}{0.75\textwidth}
  \caption{Teste de Kruskal-Wallis nos índices utilizados na pesquisa} 
  \label{tab:krusk} 

\begin{tabular}{P{3.8cm}P{3.8cm}P{3.8cm}}
\toprule
\toprule
Índices & Valor p  do teste & H0 \\ 

\midrule

IN004 & $0,00$ & Rejeita \\ 
IN005 & $0,00$ & Rejeita \\ 
IN006 & $0,00$ & Rejeita \\ 
\hline
IN015 & $0,08$ & Rejeita \\ 
IN016 & $0,00$ & Rejeita \\ 
IN055 & $0,45$ & Aceita \\ 
\hline
IN012 & $0,42$ & Aceita \\ 
IN102 & $0,49$ & Aceita \\ 
IN083 & $0,06$ & Rejeita \\ 
\hline
IN008 & $0,49$ & Aceita \\ 
IN026 & $0,00$ & Rejeita \\ 

\bottomrule

\end{tabular} 
	\footnotesize \\
		Fonte: Elaborado pelos autores. \\ 
		Nota: A hipótese nula (H0) é de que os valores dos fazem parte da mesma população quando consideramos as diferentes naturezas jurídicas. \\
				IN004 - Tarifa média praticada,
				IN005 - Tarifa média água,
				IN006 - Tarifa média esgoto,		
				IN015 - Índice de coleta de esgoto,
				IN016 - Índice de tratamento de esgoto,
				IN055 - Índice de atendimento total de água,	
				IN012 - Indicador de desempenho financeiro,
				IN102 - Índice de produtividade de pessoal total,
				IN083 - Duração média dos serviços executados,	
				IN008 - Despesa média anual por empregado,
				IN026 - Despesa de exploração por m3 faturado.   	
\end{minipage}
\end{table} 