
\begin{table}[H] \centering 
\begin{minipage}{0.3\textwidth}
  \caption{Teste de Kruskal} 
  \label{tab:krusk} 
\begin{tabular}{@{\extracolsep{5pt}} ccc} 
\toprule
\toprule
Índices & \shortstack{Valor p \\ do teste} & H0 \\ 

\midrule

IN004 & $0,00$ & Rejeita \\ 
IN005 & $0,00$ & Rejeita \\ 
IN006 & $0,00$ & Rejeita \\ 
IN015 & $0,08$ & Rejeita \\ 
IN016 & $0,00$ & Rejeita \\ 
IN055 & $0,45$ & Aceita \\ 
IN012 & $0,42$ & Aceita \\ 
IN102 & $0,49$ & Aceita \\ 
IN083 & $0,06$ & Rejeita \\ 
IN008 & $0,49$ & Aceita \\ 
IN026 & $0,00$ & Rejeita \\ 

\bottomrule

\end{tabular} 
	\footnotesize \\
		Fonte: Elaborado pelos autores. 
		Nota: A hipótese nula (H0) é de que os valores dos índices são semelhantes entre as diferentes naturezas jurídicas.
\end{minipage}
\end{table} 