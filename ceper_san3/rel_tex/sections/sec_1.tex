
\section{Introdução}

\begin{multicols}{2}

O saneamento básico ganhou bastante atenção de pesquisadores e governos ao longo das últimas décadas. Isso devido ao fato de estar associado a saúde, bem estar e também ao meio ambiente. O saneamento adequado pode trazer muitos benefícios a população, por exemplo: melhor nutrição, higiene pessoal, e prevenção a doenças relacionadas à água contaminada. Além disso pode haver também benefícios indiretos como aumento do comércio e crescimento econômico.

A Índia por exemplo, é o segundo país mais populoso do mundo e sofre com os problemas associados a falta saneamento básico\footnote{
Dados do \href{https://data.worldbank.org/indicator/SH.STA.BASS.ZS?locations=IN}{Banco Mundial} mostram que 71\% da população indiana possui acesso a serviços de saneamento básico. No entanto, vale ressaltar que esses dados não dizem nada a respeito da qualidade e condições desse serviço.
}. Esse fato motivou a criação, por parte do governo, do programa \href{https://siwi.org/latest/the-clean-india-mission-worlds-largest-sanitation-initiative/}{Índia limpa} em que uma das ações foi construir 100 milhões de banheiros públicos.


No Brasil, no dia 15/07/2020 foi sancionada a \href{http://www.planalto.gov.br/ccivil_03/_ato2019-2022/2020/lei/l14026.htm}{Lei $N^\circ \, 14.026$} que atualiza o marco legal do saneamento básico e altera a Lei $N^\circ \, 9.984$, de julho de 2000. O principal objetivo da Lei é universalizar e qualificar a prestação dos serviços de água e esgoto do país. 

No \href{https://www.fundace.org.br/assets/uploads/_up_ceper_estudos/ceper_20210_00035.pdf}{Boletim Saneamento} de fevereiro de 2021 mostramos que o setor de saneamento no estado de São Paulo vem se aprimorando, sobretudo no
atendimento total de água e coleta e tratamento de esgoto, mas ainda carece de melhorias\footnote{Constatamos isso também no \href{https://www.fundace.org.br/assets/uploads/_up_ceper_estudos/ceper_20210_00039.pdf}{Boletim Socioeconômico} de outubro de 2021.}. 

No presente estudo também avaliamos as condições de saneamento no estado de São Paulo. Porém, iremos considerar a natureza jurídica dos prestadores de serviços, uma vez que os governos, municipais e estaduais, para tentar atender a demanda de investimentos no setor tem adotado modelos de negócios em que a participação do capital privado é de suma importância para o sucesso do projeto.

Logo, iremos avaliar os prestadores de serviços de saneamento nos municípios paulistas utilizando quatro aspectos: custo dos serviços para o consumidor final, universalização dos serviços, desempenho financeiro e produtividade e despesas.

Para cumprir nossos objetivos, na seção \ref{s2} analisamos a natureza jurídica dos prestadores de serviços de saneamento nos municípios paulistas. Na seção \ref{s3} apresentamos os índices utilizados na nossa análise e a metodologia utilizada no estudo.



\end{multicols}





  